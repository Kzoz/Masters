\documentclass[10pt,twocolumn]{conference}
\usepackage[dvipdfm]{graphicx}
\usepackage{amsmath}
\usepackage{amssymb}
\usepackage{amsxtra}
\usepackage{times}
\usepackage{color}
\usepackage{longtable}
\usepackage{footnote}
\usepackage{url}
%\setlength{\voffset}{-2.0cm}
%\setlength{\hoffset}{-1cm}

\begin{document}

\title{\bf Author Guidelines for ITC-CSCC 2020 Manuscripts}

\author{Hachiro Nagoya$^1$ and Nobunaga Owari$^2$ \\
\normalsize 
$^1$Department of Engineering, XXXX University \\
Yanagido, Gifu, Gifu 501-xxxx, Japan \\
$^2$Department of Information Science and Technology, YYYY University, \\
Ibaragabasama, Nagakute, Aichi 480-xxxx, Japan \\
E-mail : $^1$xxx@xxx-u.ac.jp, $^2$xxx@gmail.com }

\maketitle

\thispagestyle{empty}

\begin{abstract}
This document is an example of what your manuscript to ITC-CSCC 2020 should look like. The abstract should appear at the top of the left-hand column of text. The abstract should contain about 100 words, and should be identical to the abstract text submitted electronically on the conference website. All manuscripts must be in English and in black fonts.
\end{abstract}

\vspace{0.3em}

\section{Introduction}
\noindent
These guidelines include complete descriptions of the fonts, spacing, and related information for preparing the ITC-CSCC 2020 proceedings manuscripts.  The format here described allows for a graceful transition to the style required for that publication.

\section{General Instructions}
\noindent
The length of a manuscript should be {\em 4 to 6 pages}. Less than 4 and/or more than 6 page papers are not automatically submitted. All material, including text, illustrations, and charts, must be kept within a print area of 178 mm wide by 241 mm high in {\em A4 paper size}.  Do not write or print anything outside the print area.  The top and bottom margins must be 28 mm, and the left and right margins must be 16 mm.  All text must be in a two-column format. Columns are to be 86 mm wide, with a 6 mm space between them.  Text must be fully justified.

\subsection{Page title, author(s) name(s) and affiliation(s)}
\noindent
The paper title (on the first page) should begin 28 mm from the top edge of the page, centered and in Times 14-point, boldface type.  Long title should be typed on two lines without a blank line intervening. The authors' name(s) and affiliation(s) appear below the title in Times 11-points, normal type and Times 10-points normal type, respectively. Papers with multiple authors and affiliations may require two or more lines for this information.

\subsection{Type-style and fonts}
\noindent
To achieve the best rendering in the proceedings, we strongly encourage you to use {\em Times-Roman font}. In addition, this will give the proceedings a more uniform look.  Use a font that is no smaller than 9-point type throughout the paper, including figure captions.

In 9-point type font, capital letters are 2 mm high.  If you use the smallest point size, there should be no more than 3.2 lines/cm vertically.  This is a minimum spacing; 2.75 lines/cm will make the paper much more readable.  Larger type sizes require correspondingly larger vertical spacing.  Please do not use double-space in your paper.  True-Type 1 fonts are preferred.

The first paragraph in each section should not be indented, but all following paragraphs within the section should be indented as these paragraphs demonstrate.

\subsection{Sections}
\noindent
Major headings, for example, ``1. Introduction'', should appear in lower case (initial letter capitalized), bold face, centered in the column, with one blank line before, and one blank line after.

Subheadings should appear in lower case and bold face.  They should start at the left margin on a separate line.

In any headings, use heading number in order to facilitate cross references with a period (``.'') after the heading number, not a colon.

\subsection{Footnotes}
\noindent
Use footnotes sparingly (or not at all!) and place them at the bottom of the column on the page on which they are referenced. Use Times 9-point type, single-spaced. They may be numbered or referred to by asterisks or other symbols\footnote{This is how a footnote should appear}. 
Footnotes should be separated from the text by a line\footnote{Note the line separating the footnotes from the text}.

\subsection{Illustrations, graphs, and photographs}
\noindent
Illustrations must appear within the designated margins. Wide illustrations may run across both columns. If possible, position illustrations at the top of columns, rather than in the middle or at the bottom.  Provide a caption for every illustration; number each one sequentially in the form: ``Figure 1. Caption of the Figure.'' ``Table 1. Caption of the Table.'' Type the captions for figures below the figures. Type the captions for tables above the tables. 
\begin{figure}[t]
\centering
\includegraphics{fig1.png}
\caption{A figure}
\end{figure}

\subsection{References}
\noindent
Citations within the text appear in brackets as [ref. number]. Gather the full set of references together under the heading {\bf References}; place the section before any Appendices, unless they contain references. Arrange the references in alphabetical order or in order of appearance in the paper. Provide a complete citation using a consistent format. 

\subsection{Appendix}
\noindent
 Appendices, if any, directly follow the text and the references (but see section 2.6.). Letter them in sequence and provide an informative title: {\bf Appendix A Title of Appendix}.

\section{Camera Ready Manuscript}
\noindent
The length of a camera ready manuscript is {\em 4 to 6 pages}. All illustrations, references, and appendices must be accommodated within this page limit. 
Although the proceedings of ITC-CSCC 2020 will be published in a downloadable PDF file, {\em only the manuscript of 4 to 6 pages can be submitted}. 

{\em Please DO NOT put a page number in each page.} Page numbers, session numbers, and conference identification will be inserted when the paper is included in the proceedings.
If the last page of your paper is only partially filled, arrange the columns so that they are evenly balanced if possible, rather than having one long column.
It is recommended that authors use the manual approach by putting in \verb!\newpage! at the appropriate
point or \verb!\enlargethispage{-X.Ymm}! somewhere at the top of the first column of the last page where ``\texttt{X.Ymm}'' is the amount to effectively shorten the text height of the given page.

\newpage

\section{Submission Process}
\noindent
The electronic version of camera ready manuscript {\em must be transformed to PDF file} and submitted on our website (\url{http://www.ist.aichi-pu.ac.jp/lab/qua/itc-cscc2020/}) by the deadline.

Copyright Transfer must be submitted on our website.
For more information or queries about complying to the submission process, e-mail to \url{itccscc2020@gmail.com}.

\begin{thebibliography}{}
\bibitem{jones}
T.A. Jones, ``Writing a good paper,'' {\it IEEE Trans. on General
Writing}, vol. 1, no. 2, pp.1-10, May 2002.

\bibitem{hwang}
K. Hwang, {\it Computer Arithmetic}, John Wiley, 1997.

\end{thebibliography}
\appendix
\section{Styles used in the template for MS Word}
\begin{savenotes}
\begin{table}[htb]
\caption{MS Word styles}
\centering
\begin{tabular}{|p{35mm}|p{35mm}|}
\hline
 & Style \\
\hline
paper title & Title \\
\hline
author & Author \\
\hline
affiliation & Affiliation \\
\hline
abstract & Abstract \\
\hline
heading & Heading 1 \\
\hline
subheading & Heading 2 \\
\hline
1st paragraph & Body Text \\
\hline
following paragraphs & Body Text First Indent\footnote{Automatically set for new paragraphs after Body Text} \\
\hline
reference header & Reference Heading \\
\hline
reference items & Reference \\
\hline
headings in appendix & Appendix \\
\hline
\end{tabular}
\end{table}
\end{savenotes}
\end{document}
